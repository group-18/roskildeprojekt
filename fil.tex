\documentclass[12pt,a4paper,oneside]{report}

\usepackage[utf8]{inputenc}
\usepackage[T1]{fontenc}
\usepackage[danish]{babel}

\usepackage[a4paper,width=150mm,top=30mm,bottom=25mm,headheight=2cm]{geometry}

\usepackage[style=authoryear,autocite=inline]{biblatex}
\bibliography{bib/references}

\usepackage{lastpage}
\usepackage{url}
\usepackage[hidelinks]{hyperref}

\usepackage{graphicx} % Import images
\usepackage{wallpaper}
\usepackage{pgfgantt} % Gantt Charts
\usepackage{float}

\usepackage[justification=centering]{caption}
\usepackage{subcaption}

\usepackage{color}

\usepackage{listings} % Code highlight
\definecolor{javared}{rgb}{0.6,0,0}
\definecolor{javagreen}{rgb}{0.25,0.5,0.35}
\definecolor{javapurple}{rgb}{0.5,0,0.35}
\definecolor{javadocblue}{rgb}{0.25,0.35,0.75}
 
\lstset{language=Java,
    basicstyle=\ttfamily\small,
    keywordstyle=\color{javapurple}\bfseries,
    stringstyle=\color{javared},
    commentstyle=\color{javagreen},
    morecomment=[s][\color{javadocblue}]{/**}{*/},
    numbers=left,
    numberstyle=\tiny\color{black},
    stepnumber=1,
    numbersep=10pt,
    tabsize=4,
    showspaces=false,
    showstringspaces=false,
    breaklines=true,
}

% Header style
\usepackage{fancyhdr}


\pagestyle{header-footer}

\makeatletter
\renewcommand\chapter{
    \if @openright
        \cleardoublepage
    \else
        \clearpage
    \fi

    \thispagestyle{header-footer}

    \global\@topnum\z@
    \@afterindentfalse
    \secdef\@chapter\@schapter
}
\makeatother

\usepackage{titlesec}
\definecolor{dtured}{cmyk}{0,.91,.72,.23}
\newcommand{\hsp}{\hspace{10pt}}
\titleformat{\chapter}[hang]{\Huge\bfseries}{\thechapter\hsp\textcolor{dtured}{|}\hsp}{0pt}{\Huge\bfseries}

\begin{document}
\section{Roskildeprojekt gruppe 18}
Formålet med vores kommende IT-system er at producere et billetsalgssystem, hvor man kan vælge produkt (partout, enkeltdag), se pris, ændre produkt og kan producere et salg af denne.

\section{Vision}
Vores vision er i dette projekt, at fremstille et system der kan holde til, at billetter bestilles i et stort omfang, og evt. i samme tidsramme uden at gå ned.
Systemet skal være fuldendt funktionel under hele salgsperioden.
Typisk siges der at antal billetter mindst er på 100.000 stk, og dermed skal der være tilstrækkelige ressourcer til håndtering af billetterne.\\

\section{Kravspecifikationer}

\subsection{Kravsliste}

\begin{enumerate}
    \item Billetsalgssystemet skal sælge billetter til kunder.
    \item Billetsalgssystemet skal kunne udskrive billetter til kunder.
    \item Billetsalgssystemet skal kunne reservere en billet i maksimalt 15 minutter.
    \item Billetsalgssystemet skal kunne vise kundes aktuelle / tidligere køb.
    \item Billetsalgssystemet skal kunne modtage Visa/MasterCard/Dankort.
    \item Kunden skal kunne redigere billetkøbet.
    \item Kunden skal kunne vælge forskellige billetter
    \item Billetsalgssystemet skal maksimalt kunne sælge x-antal billetter.
    \item Skal kunne se antallet af solgte billetter (evt. mere statistik).
\end{enumerate}

\subsubsection{MoSCoW}

MoSCoW, er et værktøj som kan bruges til at prioritere krav.
Herunder ses et eksempel på MoSCoW, og hvad de enkelte kategorier kan indeholde.\footnote{MoSCoW tabellen der vises, er hentet fra tidligere CDIO2-rapport af samme gruppe.} \\\\

\begin{tabular}{lll}
    \textbf{Mo} &   
    "Must have"                 &
    De mest vitale krav, vi ikke kan undgå. \\

    \textbf{S}  &   
    "Should have"               & 
    Vigtige krav, som ikke er vitale. \\

    \textbf{Co} &   
    "Could have"                & 
    The 'nice-to-haves' \\

    \textbf{W}  &   
    "Won’t have (this time)"    & 
    Things that provide little to no value you can give up on \\

\end{tabular}
\\\\

\section{Use Case-diagram}
\begin{figure}[H]
    \begin{center}
        \includegraphics[width=1\textwidth]{UseCaseDiagram.png}
    \end{center}
\end{figure}

\section{Domænemodel}
\begin{figure}[H]
    \begin{center}
        \includegraphics[width=1\textwidth]{Domainmodel.png}
    \end{center}
\end{figure}

\section{Klassediagram}
\begin{figure}[H]
    \begin{center}
        \includegraphics[width=1\textwidth]{Klassediagram.png}
    \end{center}
\end{figure}

\end{document}
